\documentclass[11pt,oneside,final,wide]{mwart}

\usepackage[utf8]{inputenc}
\usepackage{polski}
\usepackage{setspace}
\usepackage{enumitem}
\usepackage{fixfoot}

\onehalfspacing

\setlength{\parindent}{0pt}
\setlist{nolistsep}

\setenumerate[2]{label=(\alph*)}
\setenumerate[3]{label=(\roman*)}

\newcommand{\parno}[1]{\S~#1}

\newcounter{para}

\newenvironment{para}{
  \refstepcounter{para}
  \par\medskip\pagebreak[3]
	{
  	\centering\textbf{\parno{\arabic{para}}}\par}
  	\nopagebreak\smallskip
  	\ignorespaces
	}

	{
  		\par
	}

\newenvironment{alphenumerate}
	{
		\begin{enumerate}[label=(\alph*)]
  	}{
  \end{enumerate}
}

\newcounter{rozdz}
\newlength{\rozdzfil}
\setlength{\rozdzfil}{6cm}

\newcommand{\rozdz}[1]
	{
		\refstepcounter{rozdz}
		\par
  		\vskip 0pt plus \rozdzfil\penalty -200\vskip 0pt plus -\rozdzfil
  		\bigskip
  		{
	  		\centering
			\large\bfseries
			Rozdział~\Roman{rozdz}\par\medskip
			#1
			\par
		}
		\smallskip
		\nopagebreak
	}


\begin{document}
%\setcounter{page}{2}
\begin{center}
	\LARGE
  Statut „Fundacji Hackerspace Kraków”
\end{center}

\rozdz{Postanowienia ogólne}



\begin{para}
  Fundacja nosi nazwę „Fundacja Hackerspace Kraków” (zwana dalej Fundacją) i została ustanowiona aktem notarialnym z dnia 1 października 2012 roku, sporządzonym przez notariusza Edwarda Drozda w Kancelarii przy ul. Rotmistrza Zbigniewa Dunin-Wąsowicza 8/5 w Krakowie (Repozytorium A nr 5656/2012), z woli następujących fundatorów indywidualnych:

  \begin{enumerate}[leftmargin=*, labelindent=1cm]
    \item{Marii Magdaleny Skrzypek}
    \item{Mirosława Ireneusza Woźniaka}
  \end{enumerate}
  zwanych dalej fundatorami i działa na podstawie przepisów obowiązującej ustawy o fundacjach.
\end{para}

\begin{para}
  Fundacja posiada osobowość prawną.
\end{para}

\begin{para}
  Fundacja ustanowiona zostaje na czas nieokreślony.
\end{para}

\begin{para}
  Fundacja działa na terytorium Rzeczypospolitej Polskiej i poza jej granicami zgodnie z prawem RP.
\end{para}

\begin{para}
  Siedzibą Fundacji jest miasto Kraków
\end{para}

\begin{para}
  Nadzór nad Fundacją sprawuje minister właściwy do spraw nauki.
\end{para}

\begin{para}
  Fundacja może ustanawiać certyfikaty, odznaki, medale honorowe i przyznawać je wraz z innymi nagrodami i wyróżnieniami, osobom fizycznym i prawnym zasłużonym dla fundacji, przyczyniającym się do realizacji celów fundacji.
\end{para}

\rozdz{Cele i sposoby realizacji celów}

\begin{para}
  Celem fundacji jest:
  \begin{enumerate}
  \item{Propagowanie zainteresowań technicznych i edukacja techniczna wśród wszystkich grup wiekowych.}
  \item{Promocja i wspieranie przedsiębiorczości.}
  \item{Promocja zatrudnienia i aktywizacja zawodowa uczniów, studentów i absolwentów, a także osób niezatrudnionych lub chcących się przekwalifikować.}
  \item{Upowszechnianie kultury wolnego oprogramowania i wolnej wiedzy.}
  \item{Skupianie pasjonatów różnych dziedzin technicznych chętnych do wspólnego poszerzania wiedzy oraz dzielenia się doświadczeniem.}
  \item{Wspieranie rozwoju techniki, wynalazczości i innowacyjności oraz rozpowszechnianie i wdrażanie nowych rozwiązań technicznych.}
  \item{Wspieranie i upowszechnianie kultury i sztuki.}
  \item{ Promocja i organizacja wolontariatu, w tym również wolontariatu pracowniczego.}
  \end{enumerate}
\end{para}

\begin{para}
 Fundacja realizuje swoje cele poprzez:
 \begin{enumerate}
 \item{Udostępnianie przestrzeni, środków i wiedzy do realizacji projektów i zainteresowań.}
 \item{Organizację szkoleń, wykładów, konferencji, warsztatów i innych form edukacji służących poszerzaniu kompetencji i wiedzy dla wszystkich grup wiekowych.}
 \item{ Wymianę doświadczeń. }
 \item{ Współpracę z innymi instytucjami, organizacjami i osobami fizycznymi w zakresie realizacji celów fundacji. }
 \item{ Działalność naukowo-badawczą. }
 \item{ Organizację wizyt studyjnych i wyjazdów. }
 \item{ Prowadzenie kampanii promocyjno-informacyjnych. }
 \item{ Organizację wystaw, wernisaży, happeningów i innych form ekspresji artystycznej i kulturowej. }
 \end{enumerate}
\end{para}

\begin{para}
  W ramach działalności nieodpłatnej pożytku publicznego Fundacja prowadzi działalność w następującym zakresie: 
  \begin{enumerate}
    \item Działalność związana z organizacją targów, wystaw i kongresów (82.30.Z)
    \item Pozostałe pozaszkolne formy edukacji, gdzie indziej niesklasyfikowane (85.59.B)
    \item Działalność wspomagająca edukację (85.60.Z)
    \item Działalność pozostałych organizacji członkowskich, gdzie indziej niesklasyfikowana (94.99.Z)
    \item Pozaszkolne formy edukacji artystycznej (85.52.Z)
  \end{enumerate}
\end{para}

  \begin{para}
    W ramach działalności odpłatnej pożytku publicznego Fundacja prowadzi działalność w następującym zakresie: 
    \begin{enumerate}
      \item Działalność związana z organizacją targów, wystaw i kongresów (82.30.Z)
      \item Pozostałe pozaszkolne formy edukacji, gdzie indziej niesklasyfikowane (85.59.B)
      \item Działalność wspomagająca edukację (85.60.Z)
      \item Działalność pozostałych organizacji członkowskich, gdzie indziej niesklasyfikowana (94.99.Z)
      \item Pozaszkolne formy edukacji artystycznej (85.52.Z)
    \end{enumerate}
  \end{para}

\rozdz{Majątek i dochody Fundacji}

\begin{para}
  Majątek Fundacji stanowi fundusz założycielski w kwocie 2100 zł (w tym 1000 zł przeznaczone na działalność gospodarczą) oraz inne mienie nabyte przez Fundację w toku działania.
\end{para}

\begin{para}
  \begin{enumerate}
    \item{ Środki na realizację celów fundacji mogą pochodzić z:}
    \begin{enumerate}
      \item{ dochodów z majątku ruchomego, nieruchomego i praw majątkowych, }
      \item{ dochodów z odpłatnej działalności pożytku publicznego, }
      \item{ dochodów z działalności gospodarczej, }
      \item{ darowizn, spadków, zapisów, }
      \item{ dotacji i subwencji osób prawnych, }
      \item{ dochodów ze zbiórek publicznych, }
      \item{ odsetek bankowych, }
      \item{ wpływów z działalności statutowej. }
    \end{enumerate}
    \item{ Cały dochód (nadwyżka przychodów nad kosztami) organizacji jest przekazywany na działalność pożytku publicznego. } % Art. 20. ust. 1 pkt. 3 - [Status OPP] - Działalność pożytku publicznego i wolontariat. Dz.U.2024.1491 t.j.
  \end{enumerate}
\end{para}

\begin{para}
  Zabrania się: % Art. 20. ust. 1 pkt.6 lit. a-d. Dz.U.2024.1491 t.j.
  \begin{enumerate}
  \item{ Udzielania pożyczek lub zabezpieczenia zobowiązań majątkiem Fundacji w stosunku do wolontariuszy,
          beneficjentów, członków organów lub pracowników oraz osób, z którymi wolontariusze, beneficjenci, członkowie
          organów oraz pracownicy Fundacji pozostają w związku małżeńskim, we wspólnym pożyciu albo w stosunku
          pokrewieństwa lub powinowactwa w linii prostej, pokrewieństwa lub powinowactwa w linii bocznej do drugiego
          stopnia albo są związani z tytułu przysposobienia, opieki lub kurateli zwanych dalej osobami bliskimi, }
  \item{ przekazywania majątku Fundacji na rzecz wolontariuszy, beneficjentów, członków organów lub pracowników oraz
          ich osób bliskich, na zasadach innych niż w stosunku do osób trzecich, w szczególności jeżeli przekazywanie to
          następuje bezpłatnie lub na preferencyjnych warunkach, }
  \item{ wykorzystywania majątku Fundacji na rzecz wolontariuszy, beneficjentów, członków organów lub pracowników oraz
          ich osób bliskich na zasadach innych niż w stosunku do osób trzecich, chyba że to wykorzystanie bezpośrednio
          wynika ze statutowego celu Fundacji, }
  \item{ zakupu towarów lub usług od podmiotów, w których uczestniczą członkowie organizacji, członkowie jej organów lub
          pracownicy oraz ich osób bliskich na zasadach innych niż w stosunku do osób trzecich lub po cenach wyższych
          niż rynkowe. }
  \end{enumerate}
\end{para}

\rozdz{Działalność gospodarcza}

\begin{para}
 \begin{enumerate}
  \item{ Fundacja może prowadzić działalność gospodarczą bezpośrednio lub poprzez wyodrębnione organizacyjnie zakłady. }
  \item{ Działalność gospodarcza może polegać na udziale w spółce handlowej lub cywilnej. }
  \item{ Działalność gospodarcza Fundacji może być prowadzona wyłącznie w rozmiarach służących realizacji celów statutowych. }
  \item{ Działalność gospodarcza jest wyłącznie działalnością dodatkową w stosunku do działalności pożytku publicznego Fundacji. } % Art. 20. ust. 1 pkt. 2 Dz.U.2024.1491 t.j.
 \end{enumerate}
\end{para}

\begin{para}
 Fundacja może prowadzić działalność gospodarczą w następującym zakresie:
 \begin{enumerate}
  \item{ Działalność związana z zarządzaniem urządzeniami informatycznymi (62.03.Z) }
  \item{ Badania naukowe i prace rozwojowe w dziedzinie pozostałych nauk przyrodniczych i technicznych (72.19.Z) }
  \item{ Działalność fotograficzna (74.20.Z) }
  \item{ Artystyczna i literacka działalność twórcza (90.03.Z) }
  \item{ Obróbka mechaniczna elementów metalowych (25.62.Z)}
  \item{ Naprawa i konserwacja urządzeń elektronicznych i optycznych (33.13.Z)}
  \item{ Naprawa i konserwacja urządzeń elektrycznych (33.14.Z)}
  \item{ Wykonywanie instalacji elektrycznych (43.21.Z)}
  \item{ Pozostała działalność wydawnicza (58.19.Z)}
  \item{ Działalność związana z produkcją filmów, nagrań wideo i programów telewizyjnych (59.11.Z)}
  \item{ Działalność związana z oprogramowaniem (62.01.Z)}
  \item{ Działalność związana z doradztwem w zakresie informatyki (62.02.Z) }
  \item{ Pozostała działalność usługowa w zakresie technologii informatycznych i komputerowych (62.09.Z)}
  \item{ Działalność w zakresie specjalistycznego projektowania (74.10.Z)}
  \item{ Naprawa i konserwacja komputerów i urządzeń peryferyjnych (95.11.Z)}
  \item{ Naprawa i konserwacja sprzętu (tele)komunikacyjnego (95.12.Z)}
  \item{ Naprawa i konserwacja elektronicznego sprzętu powszechnego użytku (95.21.Z)}
  \item{ Pozostała działalność usługowa, gdzie indziej niesklasyfikowana (96.09.Z)}
 \end{enumerate}
\end{para}

\rozdz{Władze fundacji}

\begin{para}
  Władzami fundacji są:
  \begin{enumerate}
  \item{ Zarząd Fundacji }
  \item{ Rada fundacji }
 \end{enumerate}
\end{para}

\begin{para}\label{itemZ1}
 \begin{enumerate}
  \item{ Zarząd Fundacji ustanawiany jest na czas nieokreślony. Zarząd stanowią dwie do pięciu osób. }\label{itemZR1}
  \item{ Członkiem Zarządu może zostać osoba, która nie była skazana prawomocnym wyrokiem za przestępstwo umyślne ścigane z oskarżenia publicznego lub przestępstwo skarbowe. }\label{itemZ2} % Art. 20. ust. 1 pkt. 5 Dz.U.2024.1491 t.j.
  \item{ Pierwszych członków Zarządu powołują fundatorzy. }
  \item{ Następnych członków Zarządu na miejsce osób, które przestały pełnić tę funkcję lub dla rozszerzenia składu powołuje swą decyzją Rada. }
  \item{ W skład Zarządu mogą wchodzić fundatorzy. }
  \item{ Członkostwo w Zarządzie wygasa na skutek śmierci, choroby powodującej trwałą niezdolność do sprawowania funkcji, popełnienia przestępstwa o którym mowa w §\ref{itemZ1} pkt.\ref{itemZ2}, utraty praw publicznych, zrzeczenia się członkostwa lub odwołania przez Radę Fundacji.  }
  \item{ Zarząd podejmuje decyzje w formie uchwał, zwykłą większością głosów w obecności co najmniej połowy składu Zarządu, chyba że dalsze postanowienia statutu stanowią inaczej. }
  \item{ W sytuacji gdy Zarząd stanowią dwie osoby, decyzje podejmowane są jednomyślnie w obecności obu tych osób. }
  \item{ Tryb działania Zarządu Fundacji określa uchwalony przez niego Regulamin. }
  \item{ Do reprezentacji Fundacji i zaciągania zobowiązań jej imieniem do kwoty 5000 zł, upoważniony jest każdy z członków Zarządu samodzielnie. Do zaciągania zobowiązań jej imieniem powyżej kwoty 5000 zł, dwóch członków Zarządu działających łącznie.}
  \item{ Członkowie Zarządu mogą otrzymywać wynagrodzenie o ile Zarząd podejmie stosowną uchwałę o przyznawaniu i wysokości wynagrodzenia. }
  \item{ Członkowie Zarządu mogą pozostawać z Fundacją w stosunku pracy. }
  \item{ Do zakresu działania Zarządu Fundacji należy podejmowanie decyzji we wszystkich sprawach nie przekazanych do kompetencji innym organom lub jednostkom organizacyjnym a w szczególności:
   \begin{enumerate}
    \item{ kierowanie działalnością Fundacji }
    \item{ sprawowanie Zarządu nad majątkiem Fundacji }
    \item{ reprezentowanie Fundacji na zewnątrz }
    \item{ organizowanie i nadzorowanie działalności gospodarczej Fundacji }
    \item{ tworzenie, likwidacja i nadzorowanie jednostek organizacyjnych oraz ustalanie zakresu ich działalności }
    \item{ uchwalanie regulaminów }
    \item{ przygotowywanie rocznych sprawozdań z działalności oraz sprawozdań finansowych }
   \end{enumerate}
  }
 \end{enumerate}
\end{para}

\begin{para}\label{itemR1}
 \begin{enumerate}
  \item{ Rada składa się z co najmniej trzech osób powoływanych na czas nieokreślony.}
  \item{ Członkowie Rady: }\label{itemR2}
  \begin{enumerate} % Art. 20. ust. 1 pkt.4 lit. a-c. Dz.U.2024.1491 t.j.
    \item{ nie mogą być członkami organu zarządzającego ani pozostawać z nimi w związku małżeńskim, we wspólnym pożyciu, w stosunku pokrewieństwa, powinowactwa lub podległości służbowej, }\label{itemR3}
    \item{ nie byli skazani prawomocnym wyrokiem za przestępstwo umyślne ścigane z oskarżenia publicznego lub przestępstwo skarbowe, }\label{itemR4}
    \item{ mogą otrzymywać z tytułu pełnienia funkcji w takim organie zwrot uzasadnionych kosztów lub wynagrodzenie w wysokości 
    nie wyższej niż przeciętne miesięczne wynagrodzenie w sektorze przedsiębiorstw ogłoszone przez Prezesa Głównego Urzędu Statystycznego za rok poprzedni.}
  \end{enumerate}
  \item{ Pierwszych członków Rady powołują fundatorzy. }
  \item{ Następnych członków Rady na miejsce osób, które przestały pełnić tę funkcję lub dla rozszerzenia składu powołuje swą decyzją Rada.  }
  \item{ Członkostwo w Radzie wygasa, na skutek śmierci, choroby powodującej trwałą niezdolność do sprawowania funkcji, wystąpienia przesłanek spełniących §\ref{itemR1} pkt.\ref{itemR2} lit.\ref{itemR3},
   popełnienia przestępstwa o którym mowa w §\ref{itemR1} pkt.\ref{itemR2} lit.\ref{itemR4}, utraty praw publicznych, zrzeczenia się członkostwa lub odwołania decyzją Rady.  }
  \item{ Rada w drodze uchwały podczas pierwszego posiedzenia wybiera ze swego grona przewodniczącego Rady. Przewodniczący Rady kieruje pracami Rady, reprezentuje ją na zewnątrz oraz zwołuje i przewodniczy zebraniom Rady. }
  \item{ O posiedzeniu Rady muszą być powiadomieni wszyscy członkowie Rady. }
  \item{ Tryb działania Rady Fundacji określa uchwalony przez nią regulamin. }
  \item{ Rada podejmuje decyzje w formie uchwał zwykłą większością głosów w obecności co najmniej połowy składu Rady. }
   \item{ Do zadań Rady Fundacji należy:
   \begin{enumerate}
    \item{ powoływanie i odwoływanie członków Zarządu Fundacji; }
    \item{ powoływanie i odwoływanie członków Rady Fundacji; }
    \item{ opiniowanie i nadzór nad decyzjmi i działaniami Zarządu. }
   \end{enumerate}
   \item{ Rada odwołując członka Zarządu doprowadzając do niespełnienia zapisu §\ref{itemZ1} pkt.\ref{itemZR1} ma obowiązek równolegle powołać kolejne osoby tak aby zapis był spełniony. }
   }
 \end{enumerate}
\end{para}

\rozdz{Zmiana statutu}

\begin{para}
 \begin{enumerate}
  \item{ Zmiana statutu może nastąpić na drodze uchwały Zarządu Fundacji. }
  \item{ Uchwały dotyczące zmian statutu podejmowane są w obecności wszystkich członków Zarządu zwykłą większością głosów. }
  \item{ Zmiana statutu wymaga opinii Rady Fundacji. }
  \item{ Zmiana statutu może dotyczyć wszystkich zapisów umieszczonych w statucie. }
  \item{ Zmiana statutu może obejmować powoływanie nowych organów Fundacji oraz zmianę celu działalności Fundacji. }
 \end{enumerate}
\end{para}

\rozdz{Postanowienia końcowe}

\begin{para}
 \begin{enumerate}
  \item{ O likwidacji Fundacji na skutek zrealizowania jej celów lub wyczerpania środków decyduje Zarząd Fundacji. }
  \item{ Uchwała dotycząca likwidacji Fundacji podejmowana jest w obecności wszystkich członków Zarządu zwykłą większością głosów. }
  \item{ Likwidacja Fundacji wymaga opinii Rady Fundacji. }
  \item{ Jeżeli Zarząd Fundacji w uchwale, o rozwiązaniu Fundacji, nie powoła likwidatorów, likwidację Fundacji przeprowadza Zarząd. }
  \item{ Majątek pozostały po likwidacji Fundacji zostanie przekazany na cele pożytku publicznego, wskazanym przez Radę organizacjom pozarządowym działającym na terytorium Rzeczpospolitej Polskiej, których cele statutowe zbliżone są do celów Fundacji. }
 \end{enumerate}
\end{para}



\begin{para}
 \begin{enumerate}
  \item{ Fundacja może się połączyć z inną fundacją dla efektywnego realizowania swoich celów. }
  \item{ Połączenie z inną fundacją nie może nastąpić, jeżeli w jego wyniku mógłby ulec istotnej zmianie cel Fundacji. }
 \end{enumerate}
\end{para}

\vspace{2cm}
Statut przyjęty dnia 17.10.2024

Podpisano:
\begin{itemize}
  \item{Szymon Reiter}
  \item{Wiktor Przybylski}
  \item{Michał Zagórski}
\end{itemize}

\end{document}
