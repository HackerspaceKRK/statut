\documentclass[11pt,oneside,final,wide]{mwart}

\usepackage[utf8]{inputenc}
\usepackage{polski}
\usepackage{setspace}
\usepackage{enumitem}

\onehalfspacing

\setlength{\parindent}{0pt}
\setlist{nolistsep}

\setenumerate[2]{label=(\alph*)}
\setenumerate[3]{label=(\roman*)}

\newcommand{\parno}[1]{\S~#1}

\newcounter{para}

\newenvironment{para}{
  \refstepcounter{para}
  \par\medskip\pagebreak[3]
	{
  	\centering\textbf{\parno{\arabic{para}}}\par}
  	\nopagebreak\smallskip
  	\ignorespaces
	}

	{
  		\par
	}

\newenvironment{alphenumerate}
	{
		\begin{enumerate}[label=(\alph*)]
  	}{
  \end{enumerate}
}

\newcounter{rozdz}
\newlength{\rozdzfil}
\setlength{\rozdzfil}{6cm}

\newcommand{\rozdz}[1]
	{
		\refstepcounter{rozdz}
		\par
  		\vskip 0pt plus \rozdzfil\penalty -200\vskip 0pt plus -\rozdzfil
  		\bigskip
  		{
	  		\centering
			\large\bfseries
			Rozdział~\Roman{rozdz}\par\medskip
			#1
			\par
		}
		\smallskip
		\nopagebreak
	}

\begin{document}

\begin{center}
	\LARGE
  Statut Fundacji „Hackerspace Kraków”
\end{center}

\rozdz{Postanowienia ogólne}



\begin{para}
  Fundacja nosi nazwę „Hackerspace Kraków” (zwana dalej Fundacją) i została ustanowiona aktem notarialnym z dnia 1 października 2012 roku, sporządzonym przez notariusza Edwarda Drozda w Kancelarii przy ul. Rotmistrza Zbigniewa Dunin-Wąsowicza 8/5 w Krakowie (Repozytorium A nr 5656/2012), z woli następujących fundatorów indywidualnych:

  \begin{enumerate}[leftmargin=*, labelindent=1cm]
    \item{Marii Magdaleny Skrzypek}
    \item{Mirosława Ireneusza Woźniaka}
  \end{enumerate}
  zwanych dalej fundatorami i działa na podstawie przepisów obowiązującej ustawy o fundacjach.
\end{para}

\begin{para}
   Fundacja posiada osobowość prawną.
\end{para}

\begin{para}
  Fundacja ustanowiona zostaje na czas nieokreślony.
\end{para}

\begin{para}
  Fundacja działa na terytorium Rzeczypospolitej Polskiej i poza jej granicami zgodnie z prawem RP.
\end{para}

\begin{para}
  Siedzibą Fundacji jest miasto Kraków \footnotemark[1] .
\end{para}

\begin{para}
  Nadzór nad Fundacją sprawuje minister właściwy do spraw nauki.
\end{para}

\begin{para}
  Nadzór nad Fundacją sprawuje minister właściwy do spraw nauki.
\end{para}

\begin{para}
  Fundacja może ustanawiać certyfikaty, odznaki, medale honorowe i przyznawać je wraz z innymi nagrodami i wyróżnieniami, osobom fizycznym i prawnym zasłużonym dla fundacji, przyczyniającym się do realizacji celów fundacji.
\end{para}

\rozdz{Cele i sposoby realizacji celów}

\begin{para}
  Celem findacji jest:
  \begin{enumerate}
  \item{Propagowanie zainteresowań technicznych i edukacja techniczna wśród wszystkich grup wiekowych.}
  \item{Promocja i wspieranie przedsiębiorczości.}
  \item{Promocja zatrudnienia i aktywizacja zawodowa uczniów, studentów i absolwentów, a także osób niezatrudnionych lub chcących się przekwalifikować.}
  \item{Upowszechnianie kultury wolnego oprogramowania i wolnej wiedzy.}
  \item{Skupianie pasjonatów różnych dziedzin technicznych chętnych do wspólnego poszerzania wiedzy oraz dzielenia się doświadczeniem.}
  \item{Wspieranie rozwoju techniki, wynalazczości i innowacyjności oraz rozpowszechnianie i wdrażanie nowych rozwiązań technicznych.}
  \item{Wspieranie i upowszechnianie kultury i sztuki.}
  \item{ Promocja i organizacja wolontariatu, w tym również wolontariatu pracowniczego.}
  \end{enumerate}
\end{para}

\begin{para}
 Fundacja realizuje swoje cele poprzez:
 \begin{enumerate}
 \item{Udostępnianie przestrzeni, środków i wiedzy do realizacji projektów i zainteresowań.}
 \item{Organizację szkoleń, wykładów, konferencji, warsztatów i innych form edukacji służących poszerzaniu kompetencji i wiedzy dla wszystkich grup wiekowych.}
 \item{ Wymianę doświadczeń. }
 \item{ Współpracę z innymi instytucjami, organizacjami i osobami fizycznymi w zakresie realizacji celów fundacji. }
 \item{ Działalność naukowo-badawczą. }
 \item{ Organizację wizyt studyjnych i wyjazdów. }
 \item{ Prowadzenie kampanii promocyjno-informacyjnych. }
 \item{ Organizację wystaw, wernisaży, happeningów i innych form ekspresji artystycznej i kulturowej. }
 \end{enumerate}
\end{para}

\rozdz{Majątek i dochody Fundacji}

\begin{para}
  Majątek Fundacji stanowi fundusz założycielski w kwocie 2100 zł (w tym 1000 zł przeznaczone na działalność gospodarczą) oraz inne mienie nabyte przez Fundację w toku działania.
\end{para}

\begin{para}
  Środki na realizację celów fundacji mogą pochodzić z:
 \begin{enumerate}
 \item{  dochodów z majątku ruchomego, nieruchomego i praw majątkowych, }
 \item{ dochodów z działalności gospodarczej, }
 \item{  darowizn, spadków, zapisów, }
 \item{ otacji i subwencji osób prawnych, }
 \item{ dochodów ze zbiórek publicznych, }
 \item{ odsetek bankowych, }
 \item{ wpływów z działalności statutowej. }
 \end{enumerate}
\end{para}

\rozdz{Działalność gospodarcza}

\begin{para}
 \begin{enumerate}
  \item{ Fundacja może prowadzić działalność gospodarczą bezpośrednio lub poprzez wyodrębnione organizacyjnie zakłady. }
  \item{ Działalność gospodarcza może polegać na udziale w spółce handlowej lub cywilnej. }
  \item{ Działalność gospodarcza Fundacji może być prowadzona wyłącznie w rozmiarach służących realizacji celów statutowych. }
 \end{enumerate}
\end{para}

\begin{para}
 Fundacja może prowadzić działalność gospodarczą w następującym zakresie:
 \begin{enumerate}
  \item{ Działalność związana z zarządzaniem urządzeniami informatycznymi (62.03.Z) }
  \item{ Badania naukowe i prace rozwojowe w dziedzinie pozostałych nauk przyrodniczych i technicznych (72.19.Z) }
  \item{ Działalność fotograficzna (74.20.Z) }
  \item{ (usunięto) \footnotemark[3] }
  \item{ (usunięto) \footnotemark[3] }
  \item{ Działalność związana z organizacją targów, wystaw i kongresów (82.30.Z) }
  \item{ Pozaszkolne formy edukacji artystycznej (85.52.Z) }
  \item{ Pozostałe pozaszkolne formy edukacji, gdzie indziej niesklasyfikowane (85.59.B) }
  \item{ Działalność wspomagająca edukację (85.60.Z) }
  \item{ Artystyczna i literacka działalność twórcza (90.03.Z) }
  \item{ Obróbka mechaniczna elementów metalowych (25.62.Z) \footnotemark[3] }
  \item{ Naprawa i konserwacja urządzeń elektronicznych i optycznych (33.13.Z) \footnotemark[3] }
  \item{ Naprawa i konserwacja urządzeń elektrycznych (33.14.Z) \footnotemark[3] }
  \item{ Wykonywanie instalacji elektrycznych (43.21.Z) \footnotemark[3] }
  \item{ Pozostała działalność wydawnicza (58.19.Z) \footnotemark[3] }
  \item{ Działalność związana z produkcją filmów, nagrań wideo i programów telewizyjnych (59.11.Z) \footnotemark[3] }
  \item{ Działalność związana z oprogramowaniem (62.01.Z) \footnotemark[3] }
  \item{ Działalność związana z doradztwem w zakresie informatyki (62.02.Z) \footnotemark[3] }
  \item{ Pozostała działalność usługowa w zakresie technologii informatycznych i komputerowych (62.09.Z) \footnotemark[3] }
  \item{ Działalność w zakresie specjalistycznego projektowania (74.10.Z) \footnotemark[3] }
  \item{ Działalność pozostałych organizacji członkowskich, gdzie indziej niesklasyfikowana (94.99.Z) \footnotemark[3] }
  \item{ Naprawa i konserwacja komputerów i urządzeń peryferyjnych (95.11.Z) \footnotemark[3] }
  \item{ Naprawa i konserwacja sprzętu (tele)komunikacyjnego (95.12.Z) \footnotemark[3] }
  \item{ Naprawa i konserwacja elektronicznego sprzętu powszechnego użytku (95.21.Z) \footnotemark[3] }
  \item{ Pozostała działalność usługowa, gdzie indziej niesklasyfikowana (96.09.Z) \footnotemark[3] }
 \end{enumerate}
\end{para}

\rozdz{Władze fundacji}

\begin{para}
  Władzami fundacji są:
  \begin{enumerate}
  \item{ Zarząd Fundacji }
  \item{ Rada fundacji }
 \end{enumerate}
\end{para}

\begin{para}
 \begin{enumerate}
  \item{ Zarząd Fundacji ustanawiany jest na czas nieokreślony. Zarząd stanowią dwie do pięciu osób. \footnotemark[2] }
  \item{ Pierwszych członków Zarządu powołują fundatorzy. }
  \item{ Następnych członków Zarządu na miejsce osób, które przestały pełnić tę funkcję lub dla rozszerzenia składu powołuje swą decyzją Rada \footnotemark[2] }
  \item{ W skład Zarządu mogą wchodzić fundatorzy. }
  \item{ Członkostwo w Zarządzie wygasa na skutek śmierci, choroby powodującej trwałą niezdolność do sprawowania funkcji, utraty praw publicznych, zrzeczenia się członkostwa lub odwołania przez Radę Fundacji \footnotemark[2] }
  \item{ Zarząd podejmuje decyzje w formie uchwał, zwykłą większością głosów w obecności co najmniej połowy składu Zarządu. Chyba że dalsze postanowienia statutu stanowią inaczej. }
  \item{ W sytuacji gdy Zarząd stanowią dwie osoby, decyzje podejmowane są jednomyślnie w obecności obu tych osób. }
  \item{ Tryb działania Zarządu Fundacji określa uchwalony przez niego Regulamin. }
  \item{ Do reprezentacji Fundacji upoważniony jest każdy członek Zarządu, natomiast do składania oświadczeń woli upoważnieni są dwaj członkowie Zarządu działający łącznie. }
  \item{ Członkowie Zarządu mogą otrzymywać wynagrodzenie o ile Zarząd podejmie stosowną uchwałę o przyznawaniu i wysokości wynagrodzenia. }
  \item{ Członkowie Zarządu mogą pozostawać z Fundacją w stosunku pracy. }
  \item{ Do zakresu działania Zarządu Fundacji należy podejmowanie decyzji we wszystkich sprawach nie przekazanych do kompetencji innym organom lub jednostkom organizacyjnym a w szczególności:
   \begin{enumerate}
    \item{ kierowanie działalnością Fundacji }
    \item{ sprawowanie zarządu nad majątkiem Fundacji }
    \item{ reprezentowanie Fundacji na zewnątrz }
    \item{ organizowanie i nadzorowanie działalności gospodarczej Fundacji }
    \item{ tworzenie, likwidacja i nadzorowanie jednostek organizacyjnych oraz ustalanie zakresu ich działalności }
    \item{ uchwalanie regulaminów }
    \item{ przygotowywanie rocznych sprawozdań z działalności oraz sprawozdań finansowych }
   \end{enumerate}
  }
 \end{enumerate}
\end{para}

\begin{para}
 \begin{enumerate}
  \item{ Rada składa się z co najmniej trzech osób powoływanych na czas nieokreślony.}
  \item{ Pierwszych członków Rady powołują fundatorzy. }
  \item{ Następnych członków Rady na miejsce osób, które przestały pełnić tę funkcję lub dla rozszerzenia składu powołuje swą decyzją Rada \footnotemark[2] }
  \item{ Członkostwo w Radzie wygasa, na skutek śmierci, choroby powodującej trwałą niezdolność do sprawowania funkcji, utraty praw publicznych, zrzeczenia się członkostwa lub odwołania decyzją Rady \footnotemark[2] }
  \item{ Rada w drodze uchwały podczas pierwszego posiedzenia wybiera ze swego grona przewodniczącego Rady. Przewodniczący Rady kieruje pracami Rady, reprezentuje ją na zewnątrz oraz zwołuje i przewodniczy zebraniom Rady. }
  \item{ O posiedzeniu Rady muszą być powiadomieni wszyscy członkowie Rady. }
  \item{ Tryb działania Rady Fundacji określa uchwalony przez nią regulamin. }
  \item{ Rada podejmuje decyzje w formie uchwał zwykłą większością głosów w obecności co najmniej połowy składu Rady. }
  \item{ Członkowie Rady Fundacji mogą pozostawać z Fundacją w stosunku pracy. }
  \item{ Do zadań Rady Fundacji należy:
   \begin{enumerate}
    \item{ powoływanie i odwoływanie członków Zarządu Fundacji; }
    \item{ powoływanie i odwoływanie członków Rady Fundacji; }
    \item{ Opiniowanie decyzji i działań Zarządu. }
   \end{enumerate}
   }
 \end{enumerate}
\end{para}

\rozdz{Zmiana statutu}

\begin{para}
 \begin{enumerate}
  \item{ Zmiana statutu może nastąpić na drodze uchwały Zarządu Fundacji. }
  \item{ Uchwały dotyczące zmian statutu podejmowane są w obecności wszystkich członków Zarządu zwykłą większością głosów. }
  \item{ Zmiana statutu wymaga opinii Rady Fundacji. \footnotemark[2] }
  \item{ Zmiana statutu może dotyczyć wszystkich zapisów umieszczonych w statucie. }
  \item{ Zmiana statutu może obejmować powoływanie nowych organów Fundacji oraz zmianę celu działalności Fundacji. }
 \end{enumerate}
\end{para}

\rozdz{Postanowienia końcowe}

\begin{para}
 \begin{enumerate}
  \item{ O likwidacji Fundacji na skutek zrealizowania jej celów lub wyczerpania środków decyduje Zarząd Fundacji. }
  \item{ Uchwała dotycząca likwidacji Fundacji podejmowana jest w obecności wszystkich członków Zarządu zwykłą większością głosów. }
  \item{ Likwidacja Fundacji wymaga opinii Rady Fundacji. \footnotemark[2] }
  \item{ Jeżeli Zarząd Fundacji w uchwale, o rozwiązaniu Fundacji, nie powoła likwidatorów, likwidację Fundacji przeprowadza Zarząd. }
  \item{ Majątek pozostały po likwidacji Fundacji zostanie przekazany na cele pożytku publicznego, wskazanym przez Radę organizacjom pozarządowym działającym na terytorium Rzeczpospolitej Polskiej, których cele statutowe zbliżone są do celów Fundacji. }
 \end{enumerate}
\end{para}



\begin{para}
 \begin{enumerate}
  \item{ Fundacja może się połączyć z inną fundacją dla efektywnego realizowania swoich celów. }
  \item{ Połączenie z inną fundacją nie może nastąpić, jeżeli w jego wyniku mógłby ulec istotnej zmianie cel Fundacji. }
 \end{enumerate}
\end{para}

\vspace{2cm}
Statut przyjęty dnia 2.10.2012

Podpisano:
\begin{itemize}
  \item{Maria Skrzypek}
  \item{Mirosław Woźniak}
\end{itemize}

\footnotetext[1]{Zmiana uchwałą zarządu z dnia 29 grudnia 2014 o zmianie siedziby fundacji}
\footnotetext[2]{Zmiana uchwałą zarządu z dnia 29 grudnia 2014 o zmianie sposobu wyboru władz i zmian Statutu}
\footnotetext[3]{Zmiana uchwałą zarządu z dnia 2 października 2018 o zmianie zakresu prowadzonej działalności gospodarczej}

\end{document}
